%%%%%%%%%%%%%%%%%%%%%%%%%%%%%%%%%%%%%%%%
% Classe do documento
%%%%%%%%%%%%%%%%%%%%%%%%%%%%%%%%%%%%%%%%

% Nós usamos a classe "UnB-CIC".  Deixe apenas uma das linhas
% abaixo não-comentada, dependendo se você for do bacharelado ou
% da licenciatura.

% Para tirar os comentários, é só mudar o comando para fazer nada.
\newcommand{\com}[1]{\textcolor{red}{#1}}%

\documentclass[bacharelado]{UnB-CIC}%
%\documentclass[mestrado,ppca]{UnB-CIC}%

%%%%%%%%%%%%%%%%%%%%%%%%%%%%%%%%%%%%%%%%
% Pacotes importados
%%%%%%%%%%%%%%%%%%%%%%%%%%%%%%%%%%%%%%%%

\usepackage[brazil,american]{babel}
\usepackage[T1]{fontenc}
\usepackage{indentfirst}
\usepackage{natbib}
\usepackage{xcolor,graphicx,url}
\usepackage[utf8]{inputenc}
\usepackage{amsmath,amssymb,amsthm}
\usepackage{footnote}
%\usepackage{minipage}
\usepackage{tablefootnote} 
\usepackage{listings}

%%%%%%%%%%%%%%%%%%%%%%%%%%%%%%%%%%%%%%%%
% Cores dos links
%%%%%%%%%%%%%%%%%%%%%%%%%%%%%%%%%%%%%%%%

% Veja o arquivos cores.tex se quiser ver que outras cores estão
% pré-definidas.  Utilizando o comando \hypersetup abaixo nós
% evitamos aquelas caixas vermelhas feias em volta dos links.

%%%%%%%%%%%%%%%%%%%%%%%%%%%%%%%%%%%%%%%%
% Cores do estilo Tango
%%%%%%%%%%%%%%%%%%%%%%%%%%%%%%%%%%%%%%%%

\definecolor{LightButter}{rgb}{0.98,0.91,0.31}
\definecolor{LightOrange}{rgb}{0.98,0.68,0.24}
\definecolor{LightChocolate}{rgb}{0.91,0.72,0.43}
\definecolor{LightChameleon}{rgb}{0.54,0.88,0.20}
\definecolor{LightSkyBlue}{rgb}{0.45,0.62,0.81}
\definecolor{LightPlum}{rgb}{0.68,0.50,0.66}
\definecolor{LightScarletRed}{rgb}{0.93,0.16,0.16}
\definecolor{Butter}{rgb}{0.93,0.86,0.25}
\definecolor{Orange}{rgb}{0.96,0.47,0.00}
\definecolor{Chocolate}{rgb}{0.75,0.49,0.07}
\definecolor{Chameleon}{rgb}{0.45,0.82,0.09}
\definecolor{SkyBlue}{rgb}{0.20,0.39,0.64}
\definecolor{Plum}{rgb}{0.46,0.31,0.48}
\definecolor{ScarletRed}{rgb}{0.80,0.00,0.00}
\definecolor{DarkButter}{rgb}{0.77,0.62,0.00}
\definecolor{DarkOrange}{rgb}{0.80,0.36,0.00}
\definecolor{DarkChocolate}{rgb}{0.56,0.35,0.01}
\definecolor{DarkChameleon}{rgb}{0.30,0.60,0.02}
\definecolor{DarkSkyBlue}{rgb}{0.12,0.29,0.53}
\definecolor{DarkPlum}{rgb}{0.36,0.21,0.40}
\definecolor{DarkScarletRed}{rgb}{0.64,0.00,0.00}
\definecolor{Aluminium1}{rgb}{0.93,0.93,0.92}
\definecolor{Aluminium2}{rgb}{0.82,0.84,0.81}
\definecolor{Aluminium3}{rgb}{0.73,0.74,0.71}
\definecolor{Aluminium4}{rgb}{0.53,0.54,0.52}
\definecolor{Aluminium5}{rgb}{0.33,0.34,0.32}
\definecolor{Aluminium6}{rgb}{0.18,0.20,0.21}
 
\hypersetup{
  colorlinks=true,
  linkcolor=DarkScarletRed,
  citecolor=DarkScarletRed,
  filecolor=DarkScarletRed,
  urlcolor= DarkScarletRed
}

% garantir a exibição correta do símbolo na documentação original
\DeclareTextCommandDefault\textpi{\ensuremath{\pi}}%


%%%%%%%%%%%%%%%%%%%%%%%%%%%%%%%%%%%%%%%%
% Informações sobre a monografia
%%%%%%%%%%%%%%%%%%%%%%%%%%%%%%%%%%%%%%%%
\title{UnB-CIC: Uma classe em LaTeX para textos do Departamento de Ciência da Computação}%

\orientador{\prof \dr Guilherme Novaes Ramos}{CIC/UnB}
%\coorientador[a]{\prof[a] \dr[a] Coorientadora}{MAT/UnB}
\coordenador[a]{\prof[a] \dr[a] Coorde Nadora}{CIC/UnB}
\diamesano{24}{dezembro}{2014}%

\membrobanca{\prof[a] \dr[a] Membra da Banca}{MEC}
\membrobanca{\prof \dr Membro do Banco}{CIC/UnB}

\autor{Guilherme N.}{Ramos}
\CDU{004.4}

\palavraschave{LaTeX, metodologia científica}
\keywords{LaTeX, scientific method}



\graphicspath{{.}{img/}}%
\newcommand{\unbcic}{\texttt{UnB-CIC}}%
%%%%%%%%%%%%%%%%%%%%%%%%%%%%%%%%%%%%%%%%
% Texto
%%%%%%%%%%%%%%%%%%%%%%%%%%%%%%%%%%%%%%%%

\begin{document}%
    \conteudoPreTextual%

\renewcommand{\appendixname}{Anexo}


  \textual
  
  \chapter{Introdução}%
Este documento serve de exemplo da utilização da classe \unbcic\ para escrever um texto cujo objetivo é apresentar os resultados de um trabalho. A sequência de ideias apresentada deve fluir claramente, de modo que o leitor consiga compreender os principais conceitos e resultados apresentados, bem como encontrar informações sobre conceitos secundários.

\url{http://www.escritacientifica.com/}

\section{Metologia Científica}
O método científico é um conjunto de regras básicas de como proceder para produzir conhecimento, criando algo novo ou corrigindo/incrementando conhecimentos pré-existentes. Consiste em juntar evidências empíricas verificáveis baseadas na observação sistemática e controlada, geralmente resultantes de experiências ou pesquisa de campo, e analisá-las logicamente. 

\subsection{Fontes de Informação Digital}
\begin{itemize}
	\item\href{http://scholar.google.com.br/}{Google Acadêmico}
	\item\href{http://dl.acm.org/}{ACM Digital Library}
	\item\href{http://www.periodicos.capes.gov.br/}{Portal CAPES}
	\item\href{http://ieeexplore.ieee.org/Xplore/home.jsp}{IEEE Xplore}
	\item\href{http://www.sciencedirect.com/}{ScienceDirect}
	\item\href{http://link.springer.com/}{Springer Link}
\end{itemize}

\section{\LaTeX}
Eis alguns links interessantes para familiarização com \LaTeX:
\begin{itemize}
	\item \href{http://www.latex-project.org/}{\LaTeX}
	\item \href{http://latexbr.blogspot.com.br/2010/04/introducao-ao-latex.html}{Introdução ao LaTeX}
	\item \href{https://bitbucket.org/rg3915/latex/downloads/aprendendo_latex_em_5_min.pdf}{Aprendendo \LaTeX\ em 5 minutos}
	\item \href{http://en.wikibooks.org/wiki/LaTeX}{\LaTeX\ Wikibook}
	\item \href{http://tex.stackexchange.com/}{\TeX\ - \LaTeX\ Stack Exchange}
\end{itemize}


A rede CTAN (\url{http://www.ctan.org/}), idealizada em \cite{greenwade93}, oferece
acesso milhares de contribuições do sistema.%

\section{Normas?} 
\href{http://monografias.cic.unb.br/dspace/normasGerais.pdf}{Política de Publicação de Monografias e Dissertações no Repositório Digital do CIC}%
\href{http://monografias.cic.unb.br/dspace/}{Repositório do Departamento de Ciência da Computação da UnB}

\href{http://bdm.bce.unb.br/}{Biblioteca Digital de Monografias de Graduação e Especialização}
  % inserir demais capítulos
  

  \postextual
  \bibliographystyle{plain}
  \bibliography{bibliografia}

\appendix
  \small\begin{verbatim}
% -*- mode: LaTeX; coding: utf-8; -*-
%%%%%%%%%%%%%%%%%%%%%%%%%%%%%%%%%%%%%%%%%%%%%%%%%%%%%%%%%%%%%%%%%%%%%%%%%%%%%%%
%% File    : unb-cic.cls (LaTeX2e class file)
%% Authors : Flávio Maico Vaz da Costa
%%              (based on previous versions by José Carlos L. Ralha)
%% Version : 0.96
%% Updates : 0.5  [??/11/2004] - Initial release. don't remember the day.
%%         : 0.75 [04/04/2005] - Fixed font problems, UnB logo
%%                               resolution, keywords and palavras-chave
%%                               hyphenation and generation problems,
%%                               and a few other problems.
%%         : 0.8  [08/01/2006] - Corrigido o problema causado por
%%                               bancas com quatro membros. O quarto
%%                               membro agora é OPCIONAL.
%%                               Foi criado um novo comando chamado
%%                               bibliografia. Esse comando tem dois
%%                               argumentos onde o primeiro especifica
%%                               o nome do arquivo de referencias
%%                               bibliograficas e o segundo argumento
%%                               especifica o formato. Como efeito
%%                               colateral, as referências aparecem no
%%                               sumário.
%%         : 0.9 [02/03/2008]  - Reformulação total, com nova estrutura
%%                               de opções, comandos e ambientes, adequação
%%                               do logo da UnB às normas da universidade,
%%                               inúmeras melhorias tipográficas,
%%                               aprimoramento da integração com hyperref,
%%                               melhor tratamento de erros nos comandos,
%%                               documentação e limpeza do código da classe.
%%         : 0.91 [10/05/2008] - Suporte ao XeLaTeX, aprimorado suporte para
%%                               glossaries.sty, novos comandos \capa, \CDU
%%                               e \subtitle, ajustes de margem para opções
%%                               hyperref/impressao.
%%         : 0.92 [26/05/2008] - Melhora do ambiente {definition}, suporte
%%                               a hypcap, novos comandos \fontelogo e
%%                               \slashedzero, suporte [10pt, 11pt, 12pt].
%%                               Corrigido bug de seções de apêndice quando
%%                               usando \hypersetup{bookmarksnumbered=true}.
%%         : 0.93 [09/06/2008] - Correção na contagem de páginas, valores
%%                               load e config para opção hyperref, comandos
%%                               \ifhyperref e \SetTableFigures, melhor
%%                               formatação do quadrado CIP. 
%%         : 0.94 [17/04/2014] - Inclusão da opção mpca. 
%%         : 0.95 [06/06/2014] - Remoção da opção "mpca", inclusão das opções
%%                               "doutorado", "ppginf", e "ppca" para identificar
%%                               o programa de pós-graduação. Troca do teste 
%%                               @mestrado por @posgraduacao.
%%         : 0.96 [24/06/2014] - Ajuste do nome do curso/nome do programa.
%%
\end{verbatim}

\end{document}
