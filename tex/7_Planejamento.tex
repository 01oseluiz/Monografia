Para a segunda etapa do projeto de conclusão será separado em 14 semanas (média de semanas que possui um semestre).
\tabela{Tabela de planejamento da segunda parte do projeto final}{plan_table}{| c | c | c |}
{\hline
\textbf{Semana} & \textbf{Objtivo}                                                                                      & \textbf{Descrição}                                                                                                                                                                              \\ \hline
1               & Modularizar os gráficos                                                                               & \begin{tabular}[c]{@{}l@{}}Criar arquivos separados para cada\\ cálculo, bem como a configuração\\ do gráfico\end{tabular}                                                                      \\ \hline
2               & \begin{tabular}[c]{@{}l@{}}Adaptar o filtro de kalman\\ para fusão correta de ângulos\end{tabular}    & \begin{tabular}[c]{@{}l@{}}O filtro de kalman realiza média\\ ponderada simples, é necessário\\ adaptá-lo para realizar média de\\ ângulos corretamente\end{tabular}                            \\ \hline
3               & Criar modulo comparador                                                                               & \begin{tabular}[c]{@{}l@{}}Função que irá criar um gráfico\\ comparando métodos assumindo\\ que os métodos entregam os\\ mesmos tipos de dados\end{tabular}                                     \\ \hline
4               & Criar modulo comparador                                                                               &                                                                                                                                                                                                 \\ \hline
5               & \begin{tabular}[c]{@{}l@{}}Criar modulo para mensurar\\ tempo de execução de cada método\end{tabular} & \begin{tabular}[c]{@{}l@{}}Exibir ao final da execução o\\ tempo médio de execução dos\\ métodos, como conversão do dado,\\ fusão dos dados usando filtro\\ complementar ou outros\end{tabular} \\ \hline
6               & \begin{tabular}[c]{@{}l@{}}Criar modulo para mensurar tempo\\ de execução de cada método\end{tabular} &                                                                                                                                                                                                 \\ \hline
7               & Documentação e revisão dos códigos                                                                    &                                                                                                                                                                                                 \\ \hline
8               & Geração de dados e análises                                                                           &                                                                                                                                                                                                 \\ \hline
9               & Escrita da seção de fundamentação teórica                                                             &                                                                                                                                                                                                 \\ \hline
10              & Escrita da seção de fundamentação teórica                                                             &                                                                                                                                                                                                 \\ \hline
11              & Escrita da seção de análise dos dados                                                                 &                                                                                                                                                                                                 \\ \hline
12              & Escrita do Resumo, Abstract e Conclusão                                                               &                                                                                                                                                                                                 \\ \hline
13              & Criação dá apresentação e defesa                                                                      &                                                                                                                                                                                                 \\ \hline
14              & Ajustes do texto                                                                                      &                                                                                                                                                                                                 \\ \hline
}