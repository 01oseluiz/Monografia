\chapter{Introdução}%
Este documento serve de exemplo da utilização da classe \unbcic\ para escrever um texto cujo objetivo é apresentar os resultados de um trabalho. A sequência de ideias apresentada deve fluir claramente, de modo que o leitor consiga compreender os principais conceitos e resultados apresentados, bem como encontrar informações sobre conceitos secundários.

\url{http://www.escritacientifica.com/}

\section{Metologia Científica}
O método científico é um conjunto de regras básicas de como proceder para produzir conhecimento, criando algo novo ou corrigindo/incrementando conhecimentos pré-existentes. Consiste em juntar evidências empíricas verificáveis baseadas na observação sistemática e controlada, geralmente resultantes de experiências ou pesquisa de campo, e analisá-las logicamente. 

\subsection{Fontes de Informação Digital}
\begin{itemize}
	\item\href{http://scholar.google.com.br/}{Google Acadêmico}
	\item\href{http://dl.acm.org/}{ACM Digital Library}
	\item\href{http://www.periodicos.capes.gov.br/}{Portal CAPES}
	\item\href{http://ieeexplore.ieee.org/Xplore/home.jsp}{IEEE Xplore}
	\item\href{http://www.sciencedirect.com/}{ScienceDirect}
	\item\href{http://link.springer.com/}{Springer Link}
\end{itemize}

\section{\LaTeX}
Eis alguns links interessantes para familiarização com \LaTeX:
\begin{itemize}
	\item \href{http://www.latex-project.org/}{\LaTeX}
	\item \href{http://latexbr.blogspot.com.br/2010/04/introducao-ao-latex.html}{Introdução ao LaTeX}
	\item \href{https://bitbucket.org/rg3915/latex/downloads/aprendendo_latex_em_5_min.pdf}{Aprendendo \LaTeX\ em 5 minutos}
	\item \href{http://en.wikibooks.org/wiki/LaTeX}{\LaTeX\ Wikibook}
	\item \href{http://tex.stackexchange.com/}{\TeX\ - \LaTeX\ Stack Exchange}
\end{itemize}


A rede CTAN (\url{http://www.ctan.org/}), idealizada em \cite{greenwade93}, oferece
acesso milhares de contribuições do sistema.%

\section{Normas?} 
\href{http://monografias.cic.unb.br/dspace/normasGerais.pdf}{Política de Publicação de Monografias e Dissertações no Repositório Digital do CIC}%
\href{http://monografias.cic.unb.br/dspace/}{Repositório do Departamento de Ciência da Computação da UnB}

\href{http://bdm.bce.unb.br/}{Biblioteca Digital de Monografias de Graduação e Especialização}