O \emph{resumo} é um texto inaugural para quem quer conhecer o trabalho, deve conter 
uma breve descrição de todo o trabalho (apenas um parágrafo). Portanto, só deve 
ser escrito após o texto estar pronto. Não é uma coletânea de frases recortadas 
do trabalho, mas uma apresentação concisa dos pontos relevantes, de modo que o
leitor tenha uma ideia completa do que lhe espera. É seguido de três palavras-chave 
que devem indicar claramente a que se refere o seu trabalho. Neste texto, usa-se
informações úteis a produção de trabalhos científicos para descrever e exemplificar 
como utililzar a classe \LaTeX\ para gerar documentos do Departamento de Ciência 
da Computação da Universidade de Brasília.